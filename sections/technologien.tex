\section{Technologien}
 
Die effektiv verwendeten Technologie ergeben sich erst durch die Arbeit.
Folgende Liste soll einen ungefähren Eindruck ergeben, die Motivation aufzeigen und ist nicht abschliessend:

\begin{itemize}
    \item \textbf{Infrastructure as Code und Policy as Code} IaC Tool wie Terraform oder Pollumi provisionieren von Ressourcen in Cloud-Umgebugen. Ähnlich dazu besitzten die verschiedenen Cloud-Provider ihre eigenen Configuration Languages und services. Diese können als Basis für eine eigenen DSL dienen. Beispiele: Azure Sentinel, Open Policy Agent / Rego
    \item \textbf{Public Cloud / Hyperscaler} Repräsentiert durch die verschiedenen grösseren Plattformen wie AWS, Azure, (RedHat) Openshift im potentiellen SGC Umfeld dienen diese uns als Analyse-Objekte
    \item \textbf{CSPM Tools} Cloud services posture management richet unterstützt unter Anderem die Erkennung und Abwehr von Risiken in Bezug auf eine Cloud-Umgebung und die allg Einhaltung von Vorgaben. Hier exisieren von diversen Hyperscalern bereits Plattformen von denen wir profitieren können. Beispiele: Azure Defender, AWS Security Hub etc.
    \item  \textbf{CWPP Tools} Cloud workfload protection tools erkennen die workload welche in einer bspw Cloud-Umgebung vorhanden sind und führt automatisch Bewertungen durch. Tools bzw die Idee dieser Art könnten die Grundlage für unser angedachtes Framework darstellen. Beispiele: Falco, Twistlock
    \item \textbf{}
\end{itemize}