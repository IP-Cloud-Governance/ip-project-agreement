\section{Zielsetzung}

\subsection{Forschungsfragen}

% TODO Si001 anhängen

\subsubsection{Compliance-Aspekte}

\textit{Welche wichtigen Aspekte eines Schutzobjektes sind für die Compliance zu berücksichtigen?}

Die einzelnen Serviceangebote der Cloud-Provider bieten jeweils unterschiedliche Servicelevel, zum Teil auch variierend zwischen Standorten. Die wichtigsten Aspekte (von Schutzobjekten) welche in eine Complianceprüfung gehören sollen identifiziert werden.

Wir wollen erforschen, wie ein Mapping zwischen den relevanten Eigenschaften von Services in der Cloud und den Aspekten von Schutzobjekten implementiert werden kann.

% Verfügbarkeit, Skalierbarkeit, Sicherheit, CIA-> alles was so compliance sein könnte

\subsubsection{Erforschung bestehende Produkte, Konzepte, Ansätze und Frameworks}

\textit{Welche Produkte existieren bereits auf dem Markt zur cloud-übergreifenden Prüfung von Compliance?}

Im Rahmen der Lösungsfindung soll von bereits bestehenden Produkten profitiert werden. Wir wollen herausfinden, welche Produkte bereits existieren und wie diese die Complianceprüfung durchführen.
Nützliche Ansätze und Konzepte sollen hervorgehoben und für eine allfällige Eigenimplementation ggf. in Betracht gezogen werden. Hier wollen wir Produkte untersuchen welche die Complianceprüfung direkt/proprietär umsetzen, unabhängig von der dahinterliegenden Technologie.

\textit{Welche bestehenden Frameworks, Sprachen oder Konzepte sind am besten geeignet zur Umsetzung von cloud-übergreifender Compliance?}

Schutzobjekte (z.B. Applikationen) werden in der SGC cloud-nativ (via zugehörigem Deployment Code) deployed. Somit wäre der Deployment Code der richtige Ort festzulegen, welche Compliance-Regeln eingehalten werden müssen. Wir möchten hier Konzepte, Ansätze und Frameworks erforschen, welche die cloud-native Umsetzung der Compliance ermöglichen.

% Abgrenzung zu SIEM

\subsubsection{Konzeption eines Frameworks}

\textit{Welche Aspekte muss ein Framework zur cloud-übergreifende compliance von Regeln berücksichtigen und wie ist hierbei eine agnostische Anwendung durch verschiedenen Interessengruppen möglich?}

Um verschiedenen Regelwerke (Si001 etc.) einheitlich anzuwenden ist eine Art Sprache notwendig die es den verschiedenen Experten (sei es CISO, Projektverantwortlichen) erlaubt ihre Bedürfnisse auszuformulieren. Diese muss gleichzeitig durch ein System interpretiert werden können damit die entsprechenden Massnahmen ergriffen werden können.

\subsection{Nicht funktionale Anforderungen}

\textit{Auf welche Art und Weise soll der Output erbracht werden?}

\begin{table}[H]
    \centering
    \begin{tabularx}{\textwidth}{|l|l|X|}
        \hline
        \textbf{\#} & \textbf{Kategorie} & \textbf{Zielbeschreibung} \\
        \hline
        1 & \vtop{\hbox{\strut Interne Nachforschung}\hbox{\strut (Optional)}} & Bestehenden Praktiken der verschiedenen internen Interessengruppen sind bekannt \\
        \hline
        2 & Marktforschung & Recherche von bestehenden Produkten, deren Funktionalität, Gemeinsamkeiten und Deckung von Anforderungen ist erfolgt \\
        % TODO Anforderungen im sinne von wie die bestehenden policies ausgelegt / interpretiert werden \newline
        % TODO Produkte bspw CSPM, CIEM, CWPP
        \hline
        3 & Konzeptionierung & Konzept für ein Framework ist erstellt \\
        % TODO Erste Idee DSL, Labeling, offen fürs effektive enforcement, einbeziehen verschiedene SecOffs
        \hline
    \end{tabularx}
    \caption{Nicht funktionale Anforderungen}
    \label{tab:nonfunctional-requirements}
\end{table}

\subsection{Funktionale Anforderungen}

\textit{Welchen Output soll das Projekt erbringen?}

\begin{table}[H]
    \centering
    \begin{tabularx}{\textwidth}{|l|l|X|}
        \hline
        \textbf{\#} & \textbf{Kategorie} & \textbf{Zielbeschreibung} \\
        \hline
        1 & Implementation & Das Framework deckt einen Servicetyp (z.B. k8s) einer Public-Cloud Plattform ab
        \\
        \hline
        2 & Implementation & Das Framework deckt einen Servicetyp einer Public-Cloud und einer private Cloud Plattform des BIT ab
        \\
        \hline
        3 & \vtop{\hbox{\strut Agnostik}\hbox{\strut (Optional)}} & Das Framework deckt mindestens zwei Public-Cloud Plattformen und eine private Cloud Plattform ab
        % TODO Erste Idee DSL, Labeling, offen fürs effektive enforcement, einbeziehen verschiedene SecOffs
        \\
        \hline
        4 & \vtop{\hbox{\strut Implementation}\hbox{\strut (Optional)}}  & Framework kann reaktiv eingesetzt werden zur Überprüfung der Compliance
        \\
        \hline
        5 & \vtop{\hbox{\strut Implementation}\hbox{\strut (Optional)}} & Framework kann präemptiv eingesetzt werden zur Enforcierung der Compliance
        \\
        \hline
    \end{tabularx}
    \caption{Funktionale Anforderungen}
    \label{tab:functional-requirements}
\end{table}

\subsection{Meilensteine}

Die folgenden Meilensteine sind ebenfalls im 

\begin{table}[H]
    \centering
    \begin{tabularx}{\textwidth}{|l|l|X|}
        \hline
        \textbf{\#} & \textbf{Datum} & \textbf{Meilenstein} \\
        \hline
        1 & 09.03.2025 & Projektvereinbarung signiert
        \\
        \hline
        2 & xx.xx.xx & Recherche Produkte und bestehende Praktiken abgeschlossen
        % TODO Anforderungen im sinne von wie die bestehenden policies ausgelegt / interpretiert werden \newline
        % TODO Produkte bspw CSPM, CIEM, CWPP
        \\
        \hline
        3 & xx.xx.xx & erster Entwurf Konzept für Framework
        \\
        \hline
        3 & xx.xx.xx & Implementation erster PoC
        \\
        \hline
        3 & 14.08.25 & Abgabe Arbeit
        \\
        \hline
    \end{tabularx}
    \caption{Meilensteine}
    \label{tab:milestones}
\end{table}