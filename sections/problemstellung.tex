\section{Problemstellung}

Mit dem Betreiben von Services geht die Einhaltung von governance Regeln einher.

Die Sicherheitsanforderungen der Fachanwendungen entstehen oft aus Standards / Richtlinien / Normen der BV welche manuell und plattformspezifisch umgesetzt werden.
Nennenswert ist hier u.A. Si001 für den IT Grundschutz in der Bundesverwaltung welcher ein Mindestmass an Vorgaben gibt.

Die daraus resultierenden Regeln werden oft unterschiedlich definiert, abgelegt und angewendet.
Zusätzlich besteht die Gefahr, dass Standards/Richtlinien/Normen der BV unterschiedlich interpretiert werden können und somit dann andere Regeln angewendet werden.

Mit Ausblick auf das SGC Vorhaben bekommt diese Situation nochmals mehr an Bedeutung weil die Anzahl an Plattformen steigt und es somit schwieriger wird die Übersicht zu wahren.