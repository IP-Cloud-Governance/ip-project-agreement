\section{Projektorganisation}

Das ip5 Projekt läuft ab der Freigabe der Projektvereinbarung bis zum 14.08.2025. Die erbrachte Leistung sollte den erwarteten 180h pro Teammitglied entsprechen.
Wir orientieren uns hierzu an der agilen Methodik Kanban. Hierunter verstehen wir konkret, dass:
\begin{itemize}
    \item Die verschiedenen Tasks werden für jeden Projektbeteiligten sichtbar geführt (TODO hier link GithubBoard)
    \item Das Team inklusive dem PO synchen sich mindestens alle zwei Wochen um den aktuellen Stand zu tracken
    \item Die Arbeit erfolgt nicht in sprints sondern allg nach Priorität welche aus den jeweiligen Syncs hervorgeht
    \item Den Stakeholde wird alle 4 Wochen der aktuelle Stand präsentiert, bei welchem sie Ihren Input eingeben können
\end{itemize}

\subsection{Austausch}
Der Austausch inklusive den Synch wird zwischen den beteiligten erfolgt bevorzugt über den vorgesehen \href{https://teams.microsoft.com/l/channel/19%3A3bd20805200e482b809e5bf7b9294922%40thread.tacv2/25FS%20SGCWorkloadClassification?groupId=5c45c1ea-5d01-4a7b-8989-82ab37a27223&tenantId=9d1a5fc8-321e-4101-ae63-530730711ac2&ngc=true}{Teams-Kanal}, damit alle Interessierten auch passiv teilnehmen können.

\begin{table}[H]
    \centering
    \begin{tabularx}{\textwidth}{|l|X|}
        \hline
        \textbf{Objekt} & \textbf{Ort} \\
        \hline
        Projektarbeit & Die Projekt arbeit wird auf \href{https://gitlab.fhnw.ch/cloudgovernance/ip5-paper}{im ip5-paper repo} als LaTex Dokument abgelelgt \\
        \hline
        % TODO korrekten Link abglegen
        Code-Basis & Die verschiedenen Code-Bases werden unter ein separaten \href{TODO-link}{Github Organization} im Sinne des EMBAG öffentlich abgelegt  \\
        \hline
    \end{tabularx}
    \caption{Verwendete Plattformen}
    \label{tab:used-plattforms}
\end{table}


\begin{landscape}
\subsection{Zeitplan}
Das folgende Gantt-Diaramm stellt die Arbeit mit den Meilensteinen und groben Tasks da. Zu erwähnen ist hierbei, dass die detaillierten Tasks mittels Kanban-Board getracked werden.


\newcounter{myWeekNum}
\stepcounter{myWeekNum}
%
\newcommand{\myWeek}{\themyWeekNum
    \stepcounter{myWeekNum}
    \ifnum\themyWeekNum=53
         \setcounter{myWeekNum}{1}
    \else\fi
}
%
%%% Begin document
\setcounter{myWeekNum}{8}
\ganttset{%
calendar week text={\myWeek{}}%
}
%
\begin{figure}[h!bt]
\begin{center}
\begin{ganttchart}[
vgrid={*{6}{draw=none}, dotted},
x unit=.1cm,
y unit title=.6cm,
y unit chart=.6cm,
time slot format=isodate,
time slot format/start date=2025-02-17]{2025-02-17}{2025-08-18}
    \ganttset{bar height=.6}
    \gantttitlecalendar{year, month=name, week} \\
    
    % Kickoff
    \ganttgroup{Kickoff}{2025-02-17}{2025-03-07} \\
        \ganttbar{Austausch Kunde}{2025-02-17}{2025-02-28} \\
        \ganttmilestone[milestone/.append style={fill=red}]{Proj Vereinbarung signiert}{2025-02-28} \\

    % Analyse
    \ganttgroup{Analyse}{2025-03-03}{2025-03-14} \\
        \ganttbar{Interessengruppen}{2025-03-03}{2025-03-14} \\
        \ganttbar{Literatur und Recherche}{2025-03-03}{2025-03-14} \\
        \ganttmilestone[milestone/.append style={fill=red}]{Analyse dokumentiert}{2025-03-14} \\
    
    
    % Konzeptionierung / Implementierung
    \ganttgroup{Konzeptionierung}{2025-03-17}{2025-04-18} \\
        \ganttbar{Definieren DSL}{2025-03-17}{2025-04-04} \\
        \ganttbar{Erprobung DSL}{2025-04-07}{2025-04-11} \\
        \ganttmilestone[milestone/.append style={fill=red}]{First draft DSL}{2025-04-18} \\
    
    
    % CODE FREEZE
    \ganttmilestone[milestone/.append style={fill=red}]{Code freeze}{2025-07-14} \\


    \ganttgroup{Dokumentieren}{2025-07-14}{2025-08-18} \\


    \ganttbar[bar/.append style={fill=brown}]{Projektwoche}{2025-05-05}{2025-05-10} \\
    \end{ganttchart}
    \end{center}
    \caption{Time Plan}
\end{figure}
\end{landscape}